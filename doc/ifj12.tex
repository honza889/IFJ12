%%%%%%%%%%%%%%%%%%%%%%%%%%%%%%%%%%%%%%%%%%%%%%%%%%%%%%%%%%%%%%%%%%%%%%%%%%%%%%
%%%%%%%%%%%%%%%%%%%%%%%%%%%%%%%%%%%%%%%%%%%%%%%%%%%%%%%%%%%%%%%%%%%%%%%%%%%%%%
%%
%% Dokumentace k projektu pro předměty IFJ a IAL, 2012
%% Implementace interpretu imperativního jazyka IFJ12
%%
%%%%%%%%%%%%%%%%%%%%%%%%%%%%%%%%%%%%%%%%%%%%%%%%%%%%%%%%%%%%%%%%%%%%%%%%%%%%%%
%%%%%%%%%%%%%%%%%%%%%%%%%%%%%%%%%%%%%%%%%%%%%%%%%%%%%%%%%%%%%%%%%%%%%%%%%%%%%%
\documentclass[12pt,a4paper,titlepage,final]{article}

% matika
\usepackage[tbtags]{amsmath}
% cestina a fonty
\usepackage[czech]{babel}
\usepackage[utf8]{inputenc}
% balicky pro odkazy
\usepackage[bookmarksopen,colorlinks,plainpages=false,urlcolor=blue,unicode]{hyperref}
\usepackage{url}
% obrazky
\usepackage[dvipdf]{graphicx}
% velikost stranky
\usepackage[top=3.5cm, left=2.5cm, text={17cm, 24cm}, ignorefoot]{geometry}
\usepackage{lscape}

\begin{document}

%%%%%%%%%%%%%%%%%%%%%%%%%%%%%%%%%%%%%%%%%%%%%%%%%%%%%%%%%%%%%%%%%%%%%%%%%%%%%%
% titulní strana

\begin{titlepage}

% \vspace*{1cm}
\begin{figure}[!h]
  \centering
  \includegraphics[height=5cm]{img/logo.eps} \\
  Fakulta Informačních Technologií \\
  Vysoké Učení Technické v~Brně
\end{figure}

\vfill

\begin{center}
\begin{Large}
Dokumentace k projektu pro předměty IFJ a IAL\\
\end{Large}
\bigskip
\begin{Huge}
Implementace interpretu imperativního jazyka IFJ12.\\
\end{Huge}
\end{center}

\vfill

\begin{center}
\begin{Large}
\today
\end{Large}
\end{center}

\vfill

\begin{flushleft}
\begin{large}
\begin{tabular}{l}
Tým 039, varianta a/4/I
\end{tabular}
\newline
\begin{tabular}{ll}
Rozšíření: & FUNEXP, LOGOP, MINUS
\end{tabular}
\newline
\newline
\begin{tabular}{llll}
% * @author Biberle Zdeněk <xbiber00@stud.fit.vutbr.cz>
% * @author Doležal Jan    <xdolez52@stud.fit.vutbr.cz>
% * @author Fryč Martin    <xfrycm01@stud.fit.vutbr.cz>
% * @author Kalina Jan     <xkalin03@stud.fit.vutbr.cz>
% * @author Tretter Zdeněk <xtrett00@stud.fit.vutbr.cz>
Autoři: & Zdeněk Biberle, & xbiber00 & 20\% \\
        & Jan Doležal,    & xdolez52 & 20\% \\
        & Martin Fryč,    & xfrycm01 & 20\% \\
        & Jan Kalina,     & xkalin03 & 20\% \\
        & Zdeněk Tretter, & xtrett00 & 20\% \\
\end{tabular}
\end{large}
\end{flushleft}
\end{titlepage}


%%%%%%%%%%%%%%%%%%%%%%%%%%%%%%%%%%%%%%%%%%%%%%%%%%%%%%%%%%%%%%%%%%%%%%%%%%%%%%
% obsah
\pagestyle{plain}
\pagenumbering{roman}
\setcounter{page}{1}
\tableofcontents

%%%%%%%%%%%%%%%%%%%%%%%%%%%%%%%%%%%%%%%%%%%%%%%%%%%%%%%%%%%%%%%%%%%%%%%%%%%%%%
% textova zprava
\newpage
\pagestyle{plain}
\pagenumbering{arabic}
\setcounter{page}{1}

%%%%%%%%%%%%%%%%%%%%%%%%%%%%%%%%%%%%%%%%%%%%%%%%%%%%%%%%%%%%%%%%%%%%%%%%%%%%%%
\section{Úvod} \label{uvod}
%%%%%%%%%%%%%%%%%%%%%%%%%%%%%%%%%%%%%%%%%%%%%%%%%%%%%%%%%%%%%%%%%%%%%%%%%%%%%%
~ ~ ~Tato dokumentace popisuje tvorbu interpretu imperativního jazyka IFJ12. 
Popisuje, nejen způsob implementace interpretu, ale i~jeho testování a~také práci v~týmu.

%%%%%%%%%%%%%%%%%%%%%%%%%%%%%%%%%%%%%%%%%%%%%%%%%%%%%%%%%%%%%%%%%%%%%%%%%%%%%%
\section{Analýza problému a princip jeho řešení} \label{analyza}
%%%%%%%%%%%%%%%%%%%%%%%%%%%%%%%%%%%%%%%%%%%%%%%%%%%%%%%%%%%%%%%%%%%%%%%%%%%%%%                                                                             
%=============================================================================
\subsection{Rozbor zadání}
~ ~ ~Úkolem je vytvořit již zmíněný interpret imperativního jazyka IFJ12. Interpret 
musí navíc nabízet několik vestavěných funkcí a musí být napsán v jazyce C. Součástí 
zadání jsou i dílčí úkoly do předmětu IAL. Z varianty a/4/I vyplývá, že vyhledávácí 
algoritmus bude \textbf{Knuth-Moris-Prattův}, řadící algoritmus bude \textbf{Merge sort} 
a tabulka symbolů bude implementována pomocí \textbf{binárního vyhledávacího stromu}. 
%=============================================================================
\subsection{Princip řešení}
~ ~ ~Tady bude jen nějaký obecný pricnip, že LA načte token, da to Syn.A ten z toho vyrobí 
nevím co, pak to probehne Sem.A, a pak se to interpretuje. Kde a jak se používají tabulky
symbolů atd. Zkrátka nějaký hodně obecný pohled na celou strukturu interpretu. 
%=============================================================================

%%%%%%%%%%%%%%%%%%%%%%%%%%%%%%%%%%%%%%%%%%%%%%%%%%%%%%%%%%%%%%%%%%%%%%%%%%%%%%
\section{Práce v týmu} \label{prace_v_tymu}
%%%%%%%%%%%%%%%%%%%%%%%%%%%%%%%%%%%%%%%%%%%%%%%%%%%%%%%%%%%%%%%%%%%%%%%%%%%%%%
~ ~ ~K~úspěšnému dokončení interpretu nebyly potřeba jen dobré programátorské 
znalosti~a nadšení, ale bylo potřeba zvládnout i~spolupráci v pětičlenném týmu. 
Bylo potřeba najít efektivní způsob komunikace a~rozdělení práce abychom interpret
dokončili včas.
%=============================================================================
\subsection{Komunikace}
~ ~ ~Ke sdílení zdrojových souborů mezi členy týmu jsme používali systém pro 
správu verzí Git. Celotýmové schůzky proběhly dvě, jedna na začátku semestru 
a~druhá v polovině semestru. Ostatní komunikace probíhala už jen dle potřeby mezi 
jednotlivými členy pomocí některého z~IM klientů.
%=============================================================================
\subsection{Rozdělení práce}
~ ~ ~Práci jsme si v týmu rozdělovali tak, že si každý vybíral víceméně dobrovolně 
něco z~toho, co bylo zrovna potřeba udělat, tak abychom se pokud možno nikde 
nezasekli čekáním na dokončení některé z~částí potřebné pro implementaci části jiné.
\newline\newline
Výsledné rozdělení práce dopadlo přibližně takto: \medskip

\begin{tabular}{llp{12.4cm}}
Zdeněk Biberle & - & AST, syntaktický analyzátor, sémantický analyzátor\\
Jan Doležal    & - & lexikální analyzátor, ještě něco, nevim co :P\\
Martin Fryč    & - & řetězcový vyhledávací algoritmus, sémantický analyzátor\\
Jan Kalina     & - & syntaktický analyzátor, operátory, spousta dalších věcí, něco si sem dopiš\\
Zdeněk Tretter & - & vestavěné funkce, merge sort, dokumentace\\
\end{tabular}
%=============================================================================

%%%%%%%%%%%%%%%%%%%%%%%%%%%%%%%%%%%%%%%%%%%%%%%%%%%%%%%%%%%%%%%%%%%%%%%%%%%%%%
\section{Implementace} \label{implementace}
%%%%%%%%%%%%%%%%%%%%%%%%%%%%%%%%%%%%%%%%%%%%%%%%%%%%%%%%%%%%%%%%%%%%%%%%%%%%%%
%=============================================================================
\subsection{Lexikální analyzátor}
~ ~ ~Jádrem lexikálního analyzátoru je funkce \texttt{scan}, která čte lexémy ze zdrojového 
textu a převádí je na tokeny pomocí konečného automatu znázorněného v příloze 1.

Token je reprezentován strukturou \texttt{Token}. Ta obsahuje typ tokenu, data, a číslo 
řádku na kterém se token nachází. Položka data může obsahovat název symbolu, literál ze 
zdrojového souboru, operátor nebo klíčové slovo. Lexikální analyzátor nabízí syntaktickému 
analyzátoru až 4 tokeny naráz pomocí struktury \texttt{Scanner} a přidružených funkcí
\texttt{getTok}, \texttt{getTokN}, \texttt{consumeTok} a \texttt{consumeTokN}.
%=============================================================================
\subsection{Syntaktický analyzátor}
~ ~ ~bla\\ bla
%=============================================================================
\subsection{Sémantický analyzátor}
~ ~ ~bla
%=============================================================================
\subsection{Interpretace}
~ ~ ~Interpretace je prováděna rekurzivním čtením abstraktního syntaktického stromu
a~prováděním činností popisovaných jednotlivými uzly stromu. 

~ ~ ~Hlavní interpretační funkcí je \texttt{evalFunction}, jejímž účelem je 
vytvoření prostoru pro proměnné předané funkce a poté evaluace jednotlivých
příkazů této funkce, které jsou reprezentovány jako seznam struktur \texttt{Statement},
funkcí \texttt{evalStatement}, která dále předává řízení funkcím
\texttt{evalAssignment}, \texttt{evalSubstring}, \texttt{evalLoop}, 
\texttt{evalCondition} a \texttt{evalReturn} podle druhu příkazu.

~ ~ ~Tyto funkce již přímo vykonávají činnosti popsané daným příkazem 
s využitím funkce \texttt{evalExpression}, která s využitím několika
dalších funkcí (\texttt{evalOperator}, \texttt{evalBinaryOp}, 
\texttt{evalUnaryOp}, \texttt{evalConstant}, \texttt{evalVariable} a 
\texttt{evalFunctionCall}) evaluuje libovolný výraz popsaný stromem, 
jehož kořenem je struktura \texttt{Expression}.

~ ~ ~\texttt{evalCondition} a \texttt{evalLoop} taktéž využívají funkce
\texttt{evalStatement} pro provádění do nich vnořených příkazů.

%=============================================================================
\subsection{Modul ial}
\subsubsection{Vyhledávací algoritmus - Knuth-Moris-Pratt}
~ ~ ~bla
\subsubsection{Řadící algoritmus - Merge sort}
~ ~ ~Merge sort se skládá ze dvou funkcí. Funkce \texttt{mergesort} rekurzivně dělí 
řetězec na dvě poloviny, dokud jí nezbydou jen dva znaky. Tyto poloviny  předává funkci
\texttt{merge}, která oba subřetězce seřadí a~vrátí. Funkce \texttt{mergesort} skončí 
a~o~úroveň výš se seřadí dva již seřazené subřetězce do jednoho. To se opakuje dokud 
není celý řetězec seřazený.

Protože merge sort kromě prvního řazení seřazuje vždy dva už seřazené subřetězce, 
tak se ve funkci \texttt{merge} subřetězce seřazují do nového jen do doby než, 
je jeden ze subřetězců zařazen, poté se už zbytek druhého jen přilepí na konec nově 
seřazeného řetězce.  
\subsubsection{Tabulka symbolů - Binární vyhledávací strom}
~ ~ ~Jeden uzel stromu (prvek tabulky) je reprezentován strukturou \texttt{Symbol}.
Ta obsahuje název symbolu, který je zároveň klíčem, podle kterého se ve stromu vyhledává. 
Obsahuje také index symbolu v tabulce, který je kladný v případě, že symbol je 
v~lokální tabulce symbolů a~záporný v~případě, že je v~globální tabulce. Poslední dvě 
položky jsou ukazatele na menší a~větší symbol. Struktura \texttt{SymbolTable} osahuje 
ukazatel na kořen binárního vyhledávacího stromu reprezentující tabulku symbolů a~počet 
symbolů v~tabulce.\newline\newline Pro operace nad stromem jsou implementovány tyto funkce:
\medskip

\begin{tabular}{llp{11.9cm}}
\texttt{newSymbolTable}  & - & Inicializuje ukazatel na kořen na \texttt{null} a počet symbolu na 0.\\
\texttt{setNewSymbol}    & - & Prohledá strom a pokud vkládaný symbol ve stromě není, 
                               tak vytvoří pro symbol uzel a vloží ho do stromu.\\                        
\texttt{getSymbol}       & - & Prohledá strom s globalní tabulkou, pokud tam symbol nenajde,  
                               tak prohledá strom s lokální tabulkou, pokud existuje.  
                               Pokud v jedné z tabulek symbol najde, vrátí jeho index, 
                               jinak uzel se symbolem vytvoří a vloží do lokální tabulky, 
                               pokud existuje, jinak do globální tabulky a vrátí index.\\
\texttt{freeSymbolTable} & - & Uvolní prvky stromu.\\
\end{tabular}
%=============================================================================
\subsection{Rozšíření}
\subsubsection{FUNEXP}
~ ~ ~bla
\subsubsection{LOGOP}
~ ~ ~bla
\subsubsection{MINUS}
~ ~ ~bla
%=============================================================================

%%%%%%%%%%%%%%%%%%%%%%%%%%%%%%%%%%%%%%%%%%%%%%%%%%%%%%%%%%%%%%%%%%%%%%%%%%%%%%
\section{Testování} \label{testovani}
%%%%%%%%%%%%%%%%%%%%%%%%%%%%%%%%%%%%%%%%%%%%%%%%%%%%%%%%%%%%%%%%%%%%%%%%%%%%%%
~ ~ ~Testování je nedílnou součástí vývoje software. Už od začátku jsme proto začali
psát krátké testy pro každý modul, který jsme vytvořili. Tyto testy ověřovali,
zdali námi implementované funkce v modulech vrací očekávané výstupy. Ověřovali 
jsme kromě běžných případů i~různé okrajové případy. Smyslem těchto krátkých testů 
bylo zajistit, abychom během upravování modulů nezanesli do kódu nechtěné chyby.

Dalším druhem testů byly už testy samotného interpretu, napsané v~jazyce IFJ12, 
kterými jsme mohli zjišťovat, které konstrukce jazyka IFJ12 už interpretujeme 
správně a~které ještě musíme doladit. K~těmto testům jsme použili, jak zdrojové 
texty ze zadání, tak i~některé vlastní.

%%%%%%%%%%%%%%%%%%%%%%%%%%%%%%%%%%%%%%%%%%%%%%%%%%%%%%%%%%%%%%%%%%%%%%%%%%%%%%
\section{Závěr} \label{zaver}
%%%%%%%%%%%%%%%%%%%%%%%%%%%%%%%%%%%%%%%%%%%%%%%%%%%%%%%%%%%%%%%%%%%%%%%%%%%%%%
~ ~ ~Vytvořený interpret funguje tak, jak je požadováno v~zadání. Správně interpretuje 
všechny zdrojové texty ze zadání a~splňuje požadavky na formát výstupů. Dále zvládl 
správně interpretovat i~naše vlastní testy. Naše implementace interpretu také zvládá 
velmi rychlé kopírování řetězců mezi proměnnými. Také jsme implementovali navíc tři 
rozšíření, které rozšiřují funkcionalitu interpretu o~možnost použítí unárního mínus, 
úplných logických výrazů a~výrazů v~parametrech funkcí nebo použití funkcí ve výrazech.

%%%%%%%%%%%%%%%%%%%%%%%%%%%%%%%%%%%%%%%%%%%%%%%%%%%%%%%%%%%%%%%%%%%%%%%%%%%%%%
\section{Metriky kódu} \label{metriky}
%%%%%%%%%%%%%%%%%%%%%%%%%%%%%%%%%%%%%%%%%%%%%%%%%%%%%%%%%%%%%%%%%%%%%%%%%%%%%%
\begin{tabular}{ll}
  Počet souborů: & 24 \\
  Počet řádků: & 9001 \\
  Velikost spustitelného souboru: & 42 kB \\
\end{tabular}

%%%%%%%%%%%%%%%%%%%%%%%%%%%%%%%%%%%%%%%%%%%%%%%%%%%%%%%%%%%%%%%%%%%%%%%%%%%%%%
% seznam citované literatury: každá položka je definována příkazem
% \bibitem{xyz}, kde xyz je identifikátor citace (v textu použij: \cite{xyz})
%\begin{thebibliography}{1}
%
% jedna citace:
%\bibitem{kalendar}
%BLACKBURN, B.~J.; HOLFORD-STREVENS, L.: \emph{The Oxford Companion to the
%  Year}. Oxford: Oxford University Press, 1999, ISBN 0-19-214231-3.
%
%
%\end{thebibliography}
%%%%%%%%%%%%%%%%%%%%%%%%%%%%%%%%%%%%%%%%%%%%%%%%%%%%%%%%%%%%%%%%%%%%%%%%%%%%%%
% přílohy
\appendix

\newpage Tady bude automat LA

\begin{landscape} % tabulku na sirku stranky
%%%%%%%%%%%%%%%%%%%%%%%%%%%%%%%%%%%%%%%%%%%%%%%%%%%%%%%%%%%%%%%%%%%%%%%%%%%%%%
\section{Příloha: Tabulka a pravidla precedenční syntaktické analýzy} \label{precedencnitabulka}
%%%%%%%%%%%%%%%%%%%%%%%%%%%%%%%%%%%%%%%%%%%%%%%%%%%%%%%%%%%%%%%%%%%%%%%%%%%%%%

\begin{minipage}{0.83\linewidth} % obal tabulky pro poznamky pod carou
\begin{large} % vetsi velikost pisma, aby tabulka vyplnila stranku
\begin{tabular}{|l|l|l|l|l|l|l|l|l|l|l|l|l|l|l|l|l|l|l|l|l|}
\hline
&+&-&*&/&**&(&)&\textless&\textgreater&\textless=&\textgreater=&!=& ==&in&notin&and&or&not&id&\$\\
\hline
+&\textgreater&\textgreater&\textless&\textless&\textless&\textless&\textgreater&\textgreater&\textgreater&\textgreater&\textgreater&\textgreater&\textgreater&\textgreater&\textgreater&\textgreater&\textgreater&\textgreater&\textless&\textgreater\\
\hline
-&\textgreater&\textgreater&\textless&\textless&\textless&\textless&\textgreater&\textgreater&\textgreater&\textgreater&\textgreater&\textgreater&\textgreater&\textgreater&\textgreater&\textgreater&\textgreater&\textgreater&\textless&\textgreater\\
\hline
*&\textgreater&\textgreater&\textgreater&\textgreater&\textless&\textless&\textgreater&\textgreater&\textgreater&\textgreater&\textgreater&\textgreater&\textgreater&\textgreater&\textgreater&\textgreater&\textgreater&\textgreater&\textless&\textgreater\\
\hline
/&\textgreater&\textgreater&\textgreater&\textgreater&\textless&\textless&\textgreater&\textgreater&\textgreater&\textgreater&\textgreater&\textgreater&\textgreater&\textgreater&\textgreater&\textgreater&\textgreater&\textgreater&\textless&\textgreater\\
\hline
**&\textgreater&\textgreater&\textgreater&\textgreater&\textgreater&\textless&\textgreater&\textgreater&\textgreater&\textgreater&\textgreater&\textgreater&\textgreater&\textgreater&\textgreater&\textgreater&\textgreater&\textgreater&\textless&\textgreater\\
\hline
(&\textless&\textless&\textless&\textless&\textless&\textless&=&\textless&\textless&\textless&\textless&\textless&\textless&\textless&\textless&\textless&\textless&\textless&\textless&\\
\hline
)&\textgreater&\textgreater&\textgreater&\textgreater&\textgreater&&\textgreater&\textgreater&\textgreater&\textgreater&\textgreater&\textgreater&\textgreater&\textgreater&\textgreater&\textgreater&\textgreater&\textgreater&&\textgreater\\
\hline
\textless&\textless&\textless&\textless&\textless&\textless&\textless&\textgreater&\textgreater&\textgreater&\textgreater&\textgreater&\textgreater&\textgreater&\textgreater&\textgreater&\textgreater&\textgreater&\textgreater&\textless&\textgreater\\
\hline
\textgreater&\textless&\textless&\textless&\textless&\textless&\textless&\textgreater&\textgreater&\textgreater&\textgreater&\textgreater&\textgreater&\textgreater&\textgreater&\textgreater&\textgreater&\textgreater&\textgreater&\textless&\textgreater\\
\hline
\textless=&\textless&\textless&\textless&\textless&\textless&\textless&\textgreater&\textgreater&\textgreater&\textgreater&\textgreater&\textgreater&\textgreater&\textgreater&\textgreater&\textgreater&\textgreater&\textgreater&\textless&\textgreater\\
\hline
\textgreater=&\textless&\textless&\textless&\textless&\textless&\textless&\textgreater&\textgreater&\textgreater&\textgreater&\textgreater&\textgreater&\textgreater&\textgreater&\textgreater&\textgreater&\textgreater&\textgreater&\textless&\textgreater\\
\hline
!=&\textless&\textless&\textless&\textless&\textless&\textless&\textgreater&\textgreater&\textgreater&\textgreater&\textgreater&\textgreater&\textgreater&\textgreater&\textgreater&\textgreater&\textgreater&\textgreater&\textless&\textgreater\\
\hline
 ==&\textless&\textless&\textless&\textless&\textless&\textless&\textgreater&\textgreater&\textgreater&\textgreater&\textgreater&\textgreater&\textgreater&\textgreater&\textgreater&\textgreater&\textgreater&\textgreater&\textless&\textgreater\\
\hline
in&\textless&\textless&\textless&\textless&\textless&\textless&\textgreater&\textgreater&\textgreater&\textgreater&\textgreater&\textgreater&\textgreater&\textgreater&\textgreater&\textgreater&\textgreater&\textgreater&\textless&\textgreater\\
\hline
notin&\textless&\textless&\textless&\textless&\textless&\textless&\textgreater&\textgreater&\textgreater&\textgreater&\textgreater&\textgreater&\textgreater&\textgreater&\textgreater&\textgreater&\textgreater&\textgreater&\textless&\textgreater\\
\hline
and&\textless&\textless&\textless&\textless&\textless&\textless&\textgreater&\textless&\textless&\textless&\textless&\textless&\textless&\textless&\textless&\textgreater&\textgreater&\textless&\textless&\textgreater\\
\hline
or&\textless&\textless&\textless&\textless&\textless&\textless&\textgreater&\textless&\textless&\textless&\textless&\textless&\textless&\textless&\textless&\textgreater&\textgreater&\textless&\textless&\textgreater\\
\hline
not&\textless&\textless&\textless&\textless&\textless&\textless&\textgreater&\textless&\textless&\textless&\textless&\textless&\textless&\textless&\textless&\textgreater&\textgreater&\textless&\textless&\textgreater\\
\hline
id&\textgreater&\textgreater&\textgreater&\textgreater&\textgreater&*&\textgreater&\textgreater&\textgreater&\textgreater&\textgreater&\textgreater&\textgreater&\textgreater&\textgreater&\textgreater&\textgreater&\textgreater&&\textgreater\\
\hline
-&\textgreater&\textgreater&\textgreater&\textgreater&\textgreater&\textless&\textgreater&\textgreater&\textgreater&\textgreater&\textgreater&\textgreater&\textgreater&\textgreater&\textgreater&\textgreater&\textgreater&\textgreater&\textless&\textgreater\\
\hline
\$&\textless&\textless&\textless&\textless&\textless&\textless&&\textless&\textless&\textless&\textless&\textless&\textless&\textless&\textless&\textless&\textless&\textless&\textless&\\
\hline
\end{tabular}

\end{large}
\end{minipage}
\qquad
\begin{minipage}{0.17\linewidth} % obal tabulky pro poznamky pod carou
\begin{math}
G = (N,T,P,E) \\
N = \{ E \} \\
T = \{ \\ +, -, *, /, **, (, ), <, >, <=, >=, \text{!=}, \text{==}, \text{in},
        \text{notin}, \text{and}, \text{or}, \text{not}, \text{i} \\ \} \\
P = \{ \\
1: E \rightarrow E + E \\
2: E \rightarrow E - E \\
3: E \rightarrow E * E \\
4: E \rightarrow E / E \\
5: E \rightarrow E ** E \\
6: E \rightarrow E < E \\
7: E \rightarrow E > E \\
8: E \rightarrow E <= E \\
9: E \rightarrow E >= E \\
10: E \rightarrow E \text{ != } E \\
11: E \rightarrow E \text{ == } E \\
12: E \rightarrow E \text{ in } E \\
13: E \rightarrow E \text{ notin } E \\
14: E \rightarrow E \text{ and } E \\
15: E \rightarrow E \text{ or } E \\
16: E \rightarrow \text{ not } E \\
17: E \rightarrow -E \\
18: E \rightarrow (E) \\
19: E \rightarrow \text{i} \\
\} \\ \\ \\
\end{math}
\end{minipage}
\end{landscape}

\begin{landscape} % tabulku na sirku stranky
%%%%%%%%%%%%%%%%%%%%%%%%%%%%%%%%%%%%%%%%%%%%%%%%%%%%%%%%%%%%%%%%%%%%%%%%%%%%%%
\section{Příloha: Rekurzivní sestup - LL(2) tabulka} \label{rekurzivnisestup}
%%%%%%%%%%%%%%%%%%%%%%%%%%%%%%%%%%%%%%%%%%%%%%%%%%%%%%%%%%%%%%%%%%%%%%%%%%%%%%

\begin{minipage}{\linewidth} % obal tabulky pro poznamky pod carou
\begin{large}

\begin{math}
G = (N,T,R,P) \\
N = \{ P, S, P', S', L, E', E \} \\
T = \{ =, [, ], \text{ i }, \text{ if }, \text{ else }, \text{ while }, \text{ end }, \text{ function }, (, ), \text{ , }, \text{ EOL } \} \\
\end{math}

\begin{minipage}{0.2\linewidth}
\begin{math}
1.  P  \rightarrow \varepsilon \\
2.  P  \rightarrow SP \\
3.  S  \rightarrow S' \\
4.  P' \rightarrow \varepsilon \\
5.  P' \rightarrow S'P' \\
6.  S' \rightarrow id = A \text{ EOL } \\
\end{math}
\end{minipage}
\begin{minipage}{0.42\linewidth}
\begin{math}
7.  A \rightarrow E \\
8.  A \rightarrow id[E':E'] \\
9.  S' \rightarrow \text{ if } E \text{ EOL } P' \text{ end } \text{ EOL } \\
10. S' \rightarrow \text{ while } E \text{ EOL } P' \text{ end } \text{ EOL } \\
11. S' \rightarrow \text{ return } E \text{ EOL } \\
12. S  \rightarrow \text{ function } id ( E' L ) \text{ EOL } P' \text{ end } \text{ EOL } \\
\end{math}
\end{minipage}
\begin{minipage}{0.4\linewidth}
\begin{math}
13. L  \rightarrow \varepsilon \\
14. L  \rightarrow E L \\
15. E' \rightarrow E \\
16. E' \rightarrow \varepsilon \\
\\ \\
\end{math}
\end{minipage}

\input{rectab.tex}
\end{large}
\end{minipage}
\end{landscape}

\end{document}
